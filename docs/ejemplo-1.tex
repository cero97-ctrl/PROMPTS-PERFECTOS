\documentclass[11pt,a4paper]{article}
\usepackage[utf8]{inputenc}
\usepackage[spanish]{babel}
\usepackage{geometry}
\geometry{margin=2.5cm}
\usepackage{array}
\usepackage{xcolor}
\usepackage{listings}
\usepackage{hyperref}

\lstset{
    basicstyle=\ttfamily\small,
    breaklines=true,
    frame=single,
    backgroundcolor=\color{gray!10},
    extendedchars=true,
    literate={á}{{\'a}}1 {é}{{\'e}}1 {í}{{\'i}}1 {ó}{{\'o}}1 {ú}{{\'u}}1 {ñ}{{\~n}}1
}

\title{Ejemplo Práctico: Framework ROSAS}
\author{Gemini Code Assist}
\date{\today}
\begin{document}
\maketitle


Este documento presenta un ejemplo aplicado del marco \textbf{ROSAS} (Rol, Objetivo, Situación, Acción, Secuencia) para la generación de un prompt estructurado de alta fidelidad. \\

\section{Escenario del Caso}
\textbf{Contexto:} Un Gerente de Proyecto necesita evaluar los riesgos críticos antes de migrar una base de datos financiera obsoleta a la nube. \\

\section{Desglose del Prompt (Estructura ROSAS)}

\begin{table}[h]
\centering
\renewcommand{\arraystretch}{1.5}
\begin{tabular}{|p{0.45\textwidth}|p{0.45\textwidth}|}
\hline
Componente & Detalle de la Instrucción \\
\hline
\textbf{R - Rol} & Senior Technical Project Manager con certificación PMP y especialización en infraestructura Cloud (AWS) y cumplimiento normativo. \\
\hline
\textbf{O - Objetivo} & Crear una Matriz de Evaluación de Riesgos y un Plan de Contingencia (Rollback) para la migración. \\
\hline
\textbf{S - Situación} & El cliente es una Fintech sujeta a regulaciones PCI-DSS estrictas. La migración debe ejecutarse en una ventana de 48 horas (fin de semana). La tolerancia a la pérdida de datos es 0%. \\
\hline
\textbf{A - Acción} & 1. Identificar 5 riesgos técnicos y de cumplimiento.  \newline  2. Asignar probabilidad e impacto a cada uno.  \newline  3. Definir estrategias de mitigación preventiva y reactiva. \\
\hline
\textbf{S - Secuencia} & Formato de salida: Tabla Markdown para la matriz y un Resumen Ejecutivo (bullet points) para el CTO. \\
\hline
\end{tabular}
\end{table}

\vspace{0.5cm}

\section{Prompt Final (Listo para usar)}

\begin{lstlisting}
**Rol:** Actúa como un Senior Technical Project Manager experto en migraciones a AWS y cumplimiento PCI-DSS.

**Objetivo:** Generar una Matriz de Evaluación de Riesgos detallada y un Plan de Contingencia para la migración de una base de datos legacy.

**Situación:** Trabajamos para una Fintech con tolerancia cero a la pérdida de datos. La migración se realizará en una ventana crítica de 48 horas este fin de semana. El sistema actual es inestable.

**Acción Esperada:**
1. Analiza y lista los 5 riesgos más críticos (técnicos y regulatorios).
2. Evalúa la Probabilidad (Alta/Media/Baja) y el Impacto (Crítico/Alto) de cada uno.
3. Desarrolla una estrategia de mitigación específica y define el "Trigger" exacto que activaría un Rollback.

**Secuencia de Salida:**
1. Presenta la información en una tabla Markdown con las columnas: [Riesgo | Probabilidad/Impacto | Mitigación | Trigger de Rollback].
2. Finaliza con un resumen ejecutivo de 3 puntos clave dirigido al CTO para autorizar el inicio de la operación.
\end{lstlisting}

\vspace{0.5cm}
\textit{Este ejemplo demuestra cómo la especificidad en el Rol y la Situación reduce la entropía y evita respuestas genéricas sobre gestión de riesgos.} \\
\end{document}