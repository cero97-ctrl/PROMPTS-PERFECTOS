\documentclass[11pt,a4paper]{article}
\usepackage[utf8]{inputenc}
\usepackage[spanish]{babel}
\usepackage{geometry}
\geometry{margin=2.5cm}
\usepackage{array}
\usepackage{xcolor}
\usepackage{listings}
\usepackage{hyperref}

\lstset{
    basicstyle=\ttfamily\small,
    breaklines=true,
    frame=single,
    backgroundcolor=\color{gray!10},
    extendedchars=true,
    literate={á}{{\'a}}1 {é}{{\'e}}1 {í}{{\'i}}1 {ó}{{\'o}}1 {ú}{{\'u}}1 {ñ}{{\~n}}1
}

\title{Ejemplo Práctico: Refinamiento por Desplazamiento de Perspectiva}
\author{Gemini Code Assist}
\date{\today}
\begin{document}
\maketitle


Este documento ilustra cómo utilizar la técnica de \textbf{Desplazamiento de Perspectiva} para auditar y fortalecer una idea. Esta técnica consiste en pedirle a la IA que critique o mejore su propio trabajo asumiendo un rol antagónico o complementario. \\

\section{Escenario del Caso}
\textbf{Contexto:} Un equipo de producto ha diseñado una nueva funcionalidad "Gamificada" para una app bancaria. El primer borrador (generado por un rol de Marketing) es muy entusiasta pero podría ignorar riesgos regulatorios o de usuario. \\

\vspace{0.5cm}

\section{Paso 1: El Prompt Inicial (Generación)}

Primero, generamos la idea base usando el marco ROSAS estándar. \\

\begin{lstlisting}
**Rol:** Product Manager enfocado en Engagement y Gamificación.
**Objetivo:** Describir los beneficios de la nueva función "Retos de Ahorro" para la app bancaria.
**Situación:** Queremos aumentar la retención de usuarios jóvenes (Gen Z) mediante recompensas inmediatas.
**Acción:** Lista 3 beneficios clave y un slogan atractivo que fomente el uso compulsivo de la app.
\end{lstlisting}

\textit{(Supongamos que la IA genera una respuesta muy optimista sobre puntos, medallas y diversión adictiva).} \\

\vspace{0.5cm}

\section{Paso 2: El Prompt de Refinamiento (Desplazamiento de Perspectiva)}

Aquí es donde aplicamos la técnica. En lugar de pedir "mejora esto", forzamos un cambio de lente radical para encontrar fallos. \\

\begin{lstlisting}
**Instrucción de Refinamiento:**

"Perfecto. Ahora mantén esa propuesta en memoria, pero cambia completamente tu rol.

**Nuevo Rol:** Actúa como un **Chief Risk Officer (CRO)** y Auditor de Cumplimiento Normativo muy conservador.

**Acción:**
1. Critica despiadadamente la propuesta anterior. Busca riesgos de adicción al juego (ludopatía), problemas de privacidad y trivialización de las finanzas.
2. Propón 3 'Guardarraíles' o límites de seguridad que debemos implementar para mitigar estos riesgos sin matar la funcionalidad.
"
\end{lstlisting}

\vspace{0.5cm}

\section{Paso 3: Síntesis Final (El Resultado Robusto)}

\begin{lstlisting}
"Ahora, actúa como el **CEO**. Integra la creatividad de la propuesta inicial con la seguridad del reporte de riesgos. Redacta la especificación final de la funcionalidad equilibrando ambos mundos."
\end{lstlisting}

\subsection{¿Por qué funciona esto?}
El Desplazamiento de Perspectiva rompe el "sesgo de complacencia" de la IA. Al forzarla a adoptar una postura crítica o antagónica, descubres puntos ciegos que un solo prompt, por muy bueno que sea, no podría revelar. \\
\end{document}